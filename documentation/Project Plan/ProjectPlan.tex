%%%%%%%%%%%%%%%%%%%%%%%%%%%%%%%%%%%%%%%%%
% Arsclassica Article
% LaTeX Template
% Version 1.1 (1/8/17)
%
% This template has been downloaded from:
% http://www.LaTeXTemplates.com
%
% Original author:
% Lorenzo Pantieri (http://www.lorenzopantieri.net) with extensive modifications by:
% Vel (vel@latextemplates.com)
%
% License:
% CC BY-NC-SA 3.0 (http://creativecommons.org/licenses/by-nc-sa/3.0/)
%
%%%%%%%%%%%%%%%%%%%%%%%%%%%%%%%%%%%%%%%%%

%----------------------------------------------------------------------------------------
%	PACKAGES AND OTHER DOCUMENT CONFIGURATIONS
%----------------------------------------------------------------------------------------

\documentclass[
10pt, % Main document font size
a4paper, % Paper type, use 'letterpaper' for US Letter paper
oneside, % One page layout (no page indentation)
%twoside, % Two page layout (page indentation for binding and different headers)
headinclude,footinclude, % Extra spacing for the header and footer
BCOR5mm, % Binding correction
]{scrartcl}

%%%%%%%%%%%%%%%%%%%%%%%%%%%%%%%%%%%%%%%%%
% Arsclassica Article
% Structure Specification File
%
% This file has been downloaded from:
% http://www.LaTeXTemplates.com
%
% Original author:
% Lorenzo Pantieri (http://www.lorenzopantieri.net) with extensive modifications by:
% Vel (vel@latextemplates.com)
%
% License:
% CC BY-NC-SA 3.0 (http://creativecommons.org/licenses/by-nc-sa/3.0/)
%
%%%%%%%%%%%%%%%%%%%%%%%%%%%%%%%%%%%%%%%%%

%----------------------------------------------------------------------------------------
%	REQUIRED PACKAGES
%----------------------------------------------------------------------------------------

\usepackage[
nochapters, % Turn off chapters since this is an article        
beramono, % Use the Bera Mono font for monospaced text (\texttt)
eulermath,% Use the Euler font for mathematics
pdfspacing, % Makes use of pdftex’ letter spacing capabilities via the microtype package
dottedtoc % Dotted lines leading to the page numbers in the table of contents
]{classicthesis} % The layout is based on the Classic Thesis style

\usepackage{arsclassica} % Modifies the Classic Thesis package

\usepackage[T1]{fontenc} % Use 8-bit encoding that has 256 glyphs

\usepackage[utf8]{inputenc} % Required for including letters with accents

\usepackage{graphicx} % Required for including images
\graphicspath{{Figures/}} % Set the default folder for images

\usepackage{enumitem} % Required for manipulating the whitespace between and within lists

\usepackage{lipsum} % Used for inserting dummy 'Lorem ipsum' text into the template

\usepackage{subfig} % Required for creating figures with multiple parts (subfigures)

\usepackage{amsmath,amssymb,amsthm} % For including math equations, theorems, symbols, etc

\usepackage{varioref} % More descriptive referencing

%----------------------------------------------------------------------------------------
%	THEOREM STYLES
%---------------------------------------------------------------------------------------

\theoremstyle{definition} % Define theorem styles here based on the definition style (used for definitions and examples)
\newtheorem{definition}{Definition}

\theoremstyle{plain} % Define theorem styles here based on the plain style (used for theorems, lemmas, propositions)
\newtheorem{theorem}{Theorem}

\theoremstyle{remark} % Define theorem styles here based on the remark style (used for remarks and notes)

%----------------------------------------------------------------------------------------
%	HYPERLINKS
%---------------------------------------------------------------------------------------

\hypersetup{
%draft, % Uncomment to remove all links (useful for printing in black and white)
colorlinks=true, breaklinks=true, bookmarks=true,bookmarksnumbered,
urlcolor=webbrown, linkcolor=RoyalBlue, citecolor=webgreen, % Link colors
pdftitle={}, % PDF title
pdfauthor={\textcopyright}, % PDF Author
pdfsubject={}, % PDF Subject
pdfkeywords={}, % PDF Keywords
pdfcreator={pdfLaTeX}, % PDF Creator
pdfproducer={LaTeX with hyperref and ClassicThesis} % PDF producer
} % Include the structure.tex file which specified the document structure and layout

\hyphenation{Fortran hy-phen-ation} % Specify custom hyphenation points in words with dashes where you would like hyphenation to occur, or alternatively, don't put any dashes in a word to stop hyphenation altogether

%----------------------------------------------------------------------------------------
%	TITLE AND AUTHOR(S)
%----------------------------------------------------------------------------------------

\title{\normalfont\spacedallcaps{Project Plan}} % The article title

%\subtitle{Subtitle} % Uncomment to display a subtitle

\author{\spacedlowsmallcaps{CS -01 D-Enigma}} % The article author(s) - author affiliations need to be specified in the AUTHOR AFFILIATIONS block

\date{} % An optional date to appear under the author(s)

%----------------------------------------------------------------------------------------

\begin{document}

%----------------------------------------------------------------------------------------
%	HEADERS
%----------------------------------------------------------------------------------------

\renewcommand{\sectionmark}[1]{\markright{\spacedlowsmallcaps{#1}}} % The header for all pages (oneside) or for even pages (twoside)
%\renewcommand{\subsectionmark}[1]{\markright{\thesubsection~#1}} % Uncomment when using the twoside option - this modifies the header on odd pages
\lehead{\mbox{\llap{\small\thepage\kern1em\color{halfgray} \vline}\color{halfgray}\hspace{0.5em}\rightmark\hfil}} % The header style

\pagestyle{scrheadings} % Enable the headers specified in this block

%----------------------------------------------------------------------------------------
%	TABLE OF CONTENTS & LISTS OF FIGURES AND TABLES
%----------------------------------------------------------------------------------------

\maketitle % Print the title/author/date block

\setcounter{tocdepth}{2} % Set the depth of the table of contents to show sections and subsections only

\tableofcontents % Print the table of contents


\newpage
%----------------------------------------------------------------------------------------
%	ABSTRACT
%----------------------------------------------------------------------------------------

\section{Introduction}

\subsection{\textbf{ Overview}}

The purpose of this document is to tentatively define what would be done in each phase of our project development and also the tentative time schedule for each of the processes. It will act as a guideline to the team which can be followed through the various phases of software development. 



\subsection{ Project Deliverables}

The deliverables will be:
\begin{itemize}
    

\item 	Feasibility Analysis
\item 	Project Proposal 
\item 	Project Plan
\item 	System Requirements and Specifications
\item 	Software Development Life Cycle

\item 	System Test Plan
\item 	Low and High Level Design
\item 	Software Configuration Management Plan
\item 	Software Quality Assurance Plan
\item 	Risk Management and Mitigation Plan

\item 	Test Report

\item 	User Manual

\item 	Deployment Plan



\item 	Design Documents
\item 	Test Reports

\end{itemize}
 



\subsection{ Stakeholders}
\begin{itemize}
    


\item Team Members
\item Client
\item Doctors
\item Pharmacist
\item Hospitals
\item Patients( Normal Pubic)

\end{itemize}


\subsection{Assumptions, Constraints and Risks}

Assumptions :
\vspace{0.5cm}
\\●  We Believe we may not be able to deliver the final product because of less time.
\vspace{0.5cm}

\\Constraints
\vspace{0.5cm}
\\●	We are very far from our client so daily communications are not possible and the platform we intend to use is still in the developing phase.\\
\vspace{0.5cm}
\\Risks :
\vspace{0.5cm}
\\●	A major risk here is that the system may be incomplete or contain error because this is a multi-vendor platform where concurrency issues can arise.\\

\vspace{0.5cm}

 
%----------------------------------------------------------------------------------------
%	METHODS
%----------------------------------------------------------------------------------------

\section{Goals and Scope}


\subsection{Goal}
Our project aims at  making an “Amazon” like e-commerce platform of Pharmacy for our client. The website will contain a user end, a seller end and an admin end. We do have existing solution in market (1mg etc.), however most of the system are commissioned based and require approval process from third party (platform provider). This is tedious task as customer are not directly connected with pharmacy. Additionally, this is commissioned based and that is why most of the pharmacy are still not connected with online service provider or they have their local solution. So, we will provide a solution where there is no third party commission, customer will be directly able to interact with the pharmacy.


%------------------------------------------------

\subsection{Scope}
The goal of this project is to help the pharmacist to expand their business because we are providing a platform where the pharmacist would be in direct contact with customers and would not have to be give commissions.\\
This would also help aged people and others who are not able to go to the pharmacy to buy medicines and other stuff.

\section{ Organization}
\subsection{Task Division}


The roles and responsibilities have been divided among the team members taking into consideration each team member’s skills, interest and capabilities to ensure a smooth and successful project completion. The project requires several new tools, technologies and programming languages to be learnt. The following is an elaborate distribution of work among the team members :
\vspace{2cm}
\\\textbf{Project Roles and Responsibilities:}
\begin{center}
 \begin{tabular}{||c| c| c||} 
 \hline
 Name & Role & Responsibility  \\ [1.5ex] 
 \hline\hline
 1 & 6 & 87837   \\ 
 \hline
 2 & 7 & 78  \\
 \hline
 3 & 545 & 778 \\
 \hline
 4 & 545 & 18744 \\
 \hline
 5 & 88 & 788 \\ [1ex] 
 \hline
\end{tabular}
\end{center}

\subsection{ Schedule and Milestones}
\begin{table}[htbp]
\setlength\tabcolsep{1pt}
\begin{tabular}{|| p{1cm}| p{6cm}| p{4cm}| p{3cm}||}
 \hline
Serial no. & Milestones & 	Deliverables & Proposed Deadline   \\ [1.5ex] 
 \hline\hline
 1. &	Finalize Project Idea &	Project Topic &	2nd Feb\\ 
 \hline
 2. &	Feasibility Analysis and Project Proposal &	Feasibility Report and Project Proposal	& 4th Feb \\
 \hline
 3. & Plan the work to be done in 
 phases of software development	& Project Plan & 19th Feb \\
 \hline
4 &	Requirements collection through online surveys, interviews and discussions
& SRS Document & 6th Mar \\
 \hline
5 &	Tentative User Manual &	User Manual	& 6th April\\
\hline
6 &	Designing of the system &	Design &	21st March\\
\hline
7 &	Coding of modules and unit testing & Unit Tested Modules &	8th April\\
\hline
8 &	Integrate individual modules	& &	12th April\\
\hline
9 &	Testing and finalizing &	System test report &	14th April\\[1ex] 
 \hline
\end{tabular}

\end{table}

\section{Cost Estimations}
In software engineering, one of the most important factor is cost estimation. The cost is measured in Person hours. Cost of the project is due to the documentation and software to be delivered at the end of the project. As every software used is open source, so cost of these platforms are not taken into account.\vspace{0.5cm}\\ We are using basic \textit{Constructive Cost Model (COCOMO)} for Cost Estimation.\vspace{0.5cm}
\\The total LOC in the project will be in the range of 1,100-1,500. By using these values in terms KLOC (Kilo Lines of Code) and as our team consists of a mix of experienced and inexperienced individuals, so it would be a \textit{Semi-detatched} Project.\vspace{0.5cm}
By using these values, the range of the Effort and Development Time are calculated and are as follows:\vspace{0.5cm}
\\Effort = 27.37PM 37.34PM Development Time = 7.96 months 8.88 months




\section{ COMMUNICATION AND REPORTING}
\subsection{ Communication with the Client}
The team will be in constant communication with the client (Mr Pankaj from the HCL company) in order to know his requirements and inputs under each increment. If there are any changes in the requirements, modifications will be made accordingly. We would be constantly taking his reviews on the various modules we develop and change accordingly.

\subsection{ Communication within the Group}
\begin{table}[h!]
\begin{tabular}{|| p{3cm}| p{3cm}| p{3cm}| p{3cm}||} 
 \hline
Type of Communication &
Medium & Discussion & Participants\\ [1.5ex] 
 \hline\hline
Project Meetings & Face to Face &
Current Status Problems (if any),  Documentation discussion, Application Discussion& Entire Team\\
\hline
Sharing Project Data &
Google Docs, Google Group, Github &
To keep track of current progress and completed work &
Entire Team\\
\hline 
Meeting with Teaching Assistants &
SEN Lab &
Guidance on Project related issues&
Entire Team, TAs\\[1ex] 
 \hline
\end{tabular}
\end{table}
\subsection{External Communication and Reporting:}

External communication takes place in the form of  meetings with the teaching assistant assigned to our group for guidance, assistance and to solve difficulties that we might face. We also talked to our prospective clients to know their requirements as well as build a target audience to showcase our product which might be used by them.\vspace{0.5cm}
\\We took a survey of doctors, pharmacists and various patients  to assess the existing problems which our product might improve upon and also to gauge their needs and issues with the existing model of assessment process in place.
\vspace{1cm}
\\Reporting is mainly done in the form of deadline submissions, weekly viva’s and ensuring a finished product within the stipulated time.

\section{PROJECT MONITORING AND QUALITY CONTROL}
Within the group, tasks are divided among team members are in subgroups. Regular meetings will be held to keep up with the progress of each sub-group. Mistakes and suggestions are then discussed and decided upon by majority and referencing the reading materials of the course.
\vspace{0.5cm}
\\To maintain the quality control, we try to review every requirement and deliverables of the project and see if it lives upto the standard we require as well as what the client requires. This is also done by referencing with an industry grade software performing the same purpose(Subject to availability and existence).\vspace{0.5cm}
\\During the coding phase, proper commenting with well indented code is something we always aim for so that the code can be understood by others easily and the mistakes can be rectified quickly.

\end{document}